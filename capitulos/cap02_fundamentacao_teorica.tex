\chapter{Fundamentação Teórica}
\label{ch:Fundamentação_Teorica}
	\begin{resumocapitulo}
		No cenário contemporâneo, a demanda por profissionais qualificados em habilidades digitais é exponencialmente crescente. Nesse contexto, a Digital Jungle Quest emerge como uma instituição de ensino pioneira, oferecendo uma plataforma de aprendizagem gamificada destinada a auxiliar estudantes universitários e profissionais na ampliação e aprimoramento de suas competências digitais. Através de uma abordagem interativa e imersiva, a Digital Jungle Quest proporciona uma experiência educacional única, fundamentada em uma grade curricular abrangente e atualizada, que engloba diversos aspectos essenciais das tecnologias digitais contemporâneas.
	\end{resumocapitulo}

	\section{Fundamentação Teórica da Grade Curricular da Digital Jungle Quest}

\begin{figure}
\centering
\includegraphics[width=0.75\linewidth]{figuras/fundamentação.jpg}
\caption{Fundamentação Teórica}
\end{figure}

Fundamentação Teórica

A fundamentação teórica da Digital Jungle Quest é embasada em uma seleção criteriosa de materiais de referência, cuidadosamente escolhidos para fornecer uma base sólida e abrangente aos estudantes em sua jornada de aprendizado. A seguir, destacamos algumas das disciplinas fundamentais oferecidas pela instituição, juntamente com os materiais base utilizados para cada uma delas.

Programação Front-End: HTML, CSS e JavaScript

A programação front-end é um pilar fundamental no desenvolvimento de aplicações web modernas. Para proporcionar aos estudantes uma compreensão abrangente dos elementos essenciais do front-end, a Digital Jungle Quest incorpora uma variedade de recursos de alta qualidade em sua abordagem pedagógica.

HTML: A base estrutural de qualquer página web, o HTML é ensinado através de recursos como o artigo "HTML5: A Vocabulary and Associated APIs for HTML and XHTML" pelo World Wide Web Consortium (W3C), que detalha a especificação oficial do HTML5. Complementando este conhecimento, o livro "HTML and CSS: Design and Build Websites" de Jon Duckett oferece uma introdução abrangente aos fundamentos do HTML e CSS, consolidando os princípios básicos da linguagem.

CSS (Cascading Style Sheets): Para aprofundar o entendimento de CSS, são utilizados recursos como o livro "CSS: The Definitive Guide" por Eric A. Meyer e Estelle Weyl, que abrange todos os aspectos do CSS, incluindo layouts e animações. Além disso, a publicação "CSS3: Visual QuickStart Guide" de Jason Cranford Teague explora as novas funcionalidades do CSS3, garantindo uma compreensão abrangente das capacidades mais recentes da linguagem.

JavaScript: Considerado uma das linguagens de programação mais importantes no desenvolvimento web, o JavaScript é ensinado através de obras como "JavaScript: The Good Parts" de Douglas Crockford, que oferece insights sobre as melhores práticas e abordagens eficientes na linguagem. Além disso, "Eloquent JavaScript: A Modern Introduction to Programming" de Marijn Haverbeke fornece uma introdução moderna e acessível aos conceitos fundamentais do JavaScript, preparando os estudantes para enfrentar os desafios da programação front-end com confiança e competência.

IoT (Internet das Coisas) e Arquitetura de Computadores

Além das disciplinas tradicionais de programação web, a Digital Jungle Quest também oferece uma visão abrangente e prática de tópicos avançados, como IoT e Arquitetura de Computadores.

IoT: Para a disciplina de IoT, é adotado o livro "Internet of Things: A Hands-On Approach" de Arshdeep Bahga e Vijay Madisetti, que oferece uma introdução prática e baseada em projetos ao desenvolvimento de soluções IoT. Através deste recurso, os estudantes são capacitados a compreender os princípios fundamentais por trás da Internet das Coisas e a aplicar seus conhecimentos na criação de soluções inovadoras e funcionalidades conectadas.

Arquitetura de Computadores: A compreensão da arquitetura de computadores é essencial para qualquer profissional de tecnologia da informação. Para essa disciplina, é utilizado o livro "Computer Organization and Design: The Hardware/Software Interface" de David A. Patterson e John L. Hennessy, uma obra clássica que aborda os princípios fundamentais da arquitetura de computadores de forma abrangente e acessível. Através deste recurso, os estudantes desenvolvem uma compreensão sólida dos componentes essenciais de um sistema computacional e de sua interação com o software subjacente.

Considerações Finais

Em suma, a Digital Jungle Quest representa uma abordagem inovadora e abrangente no ensino de habilidades digitais, oferecendo uma plataforma de aprendizagem gamificada que combina teoria e prática de forma integrada e envolvente. Através de uma grade curricular cuidadosamente elaborada e de uma seleção criteriosa de materiais de referência, a instituição prepara os estudantes para enfrentar os desafios do mundo digital com confiança, competência e criatividade.